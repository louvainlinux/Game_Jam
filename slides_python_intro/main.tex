	%%%%%%%%%%%%%%%%%%%%%%%%%%%%%%%%%%%%%%%%%
% Beamer Presentation
% LaTeX Template
% Version 1.0 (10/11/12)
%
% This template has been downloaded from:
% http://www.LaTeXTemplates.com
%
% License:
% CC BY-NC-SA 3.0 (http://creativecommons.org/licenses/by-nc-sa/3.0/)
%
%%%%%%%%%%%%%%%%%%%%%%%%%%%%%%%%%%%%%%%%%

%----------------------------------------------------------------------------------------
%	PACKAGES AND THEMES
%----------------------------------------------------------------------------------------

\documentclass[xcolor=table]{beamer}
\mode<presentation> {

% The Beamer class comes with a number of default slide themes
% which change the colors and layouts of slides. Below this is a list
% of all the themes, uncomment each in turn to see what they look like.

%\usetheme{default}
%\usetheme{AnnArbor}
%\usetheme{Antibes}
%\usetheme{Bergen}
%\usetheme{Berkeley}
%\usetheme{Berlin}
%\usetheme{Boadilla}
\usetheme{CambridgeUS}
%\usetheme{Copenhagen}
%\usetheme{Darmstadt}
%%\usetheme{Dresden}
%\usetheme{Frankfurt}
%\usetheme{Goettingen}
%\usetheme{Hannover}
%\usetheme{Ilmenau}
%\usetheme{JuanLesPins}
%\usetheme{Luebeck}
%\usetheme{Madrid}
%\usetheme{Malmoe}
%\usetheme{Marburg}
%\usetheme{Montpellier}
%\usetheme{PaloAlto}
%%\usetheme{Pittsburgh}
%\usetheme{Rochester}
%\usetheme{Singapore}
%\usetheme{Szeged}
%\usetheme{Warsaw}

% As well as themes, the Beamer class has a number of color themes
% for any slide theme. Uncomment each of these in turn to see how it
% changes the colors of your current slide theme.
%\usecolortheme{albatross}
%\usecolortheme{beaver}
%\usecolortheme{beetle}
%\usecolortheme{crane}
%\usecolortheme{dolphin}
%\usecolortheme{dove}
%\usecolortheme{fly}
%\usecolortheme{lily}
%\usecolortheme{orchid}
%\usecolortheme{rose}
%\usecolortheme{seagull}
%\usecolortheme{seahorse}
%\usecolortheme{whale}
%\usecolortheme{wolverine}

%\setbeamertemplate{footline} % To remove the footer line in all slides uncomment this line
%\setbeamertemplate{footline}[page number] % To replace the footer line in all slides with a simple slide count uncomment this line

%\setbeamertemplate{navigation symbols}{} % To remove the navigation symbols from the bottom of all slides uncomment this line
}
\usepackage{graphicx} % Allows including images
\usepackage{booktabs} % Allows the use of \toprule, \midrule and \bottomrule in tables
%\usepackage[T1]{fontenc}
%\usepackage[latin1]{inputenc}
\usepackage{color}
\usepackage{graphics}
\usepackage[utf8]{inputenc}
\usepackage{array}
\newcolumntype{L}[1]{>{\centering\let\newline\\\arraybackslash\hspace{0pt}}m{#1}}



\usepackage{listings}
\lstset{language=Java}
\usepackage{verbatim}


%----------------------------------------------------------------------------------------
%	TITLE PAGE
%----------------------------------------------------------------------------------------

\title[Stage Python]{TechnofuturTic - Stage Python \& Raspberry Pi} % The short title appears at the bottom of every slide, the full title is only on the title page
\subtitle{Introduction}
\titlegraphic{\includegraphics[scale=0.5]{./Ressources/Logo.png}}

\author[] {\\Florence Blondiaux \\ Jean-Martin Vlaeminck} % Your name
\institute[] % Your institution as it will appear on the bottom of every slide, may be shorthand to save space
{
Université Catholique de Louvain \\ % Your institution for the title page
\medskip

}
\date{06/08/18 au 10/08/18}

\begin{document}

\begin{frame}
\titlepage % Print the title page as the first slide
\end{frame}


%------------------------------------------------



\section{Introduction}

\begin{frame}
\frametitle{Déroulement de la semaine}
    \begin{itemize}
        \item Prise de contact
        \item Initiation à Python
        \item Découverte Raspberry Pi
        \item Projet 1 : jeu Raspberry Pi
        \item Projet 2 : station météo %TODO projet ?
        \item Présentation aux parents
    \end{itemize}
\end{frame}

\section{Organisation}

\begin{frame}
\frametitle{Travail}
    \textbf{Travail en groupe}
    \begin{itemize}
        \item 2 ou 3 personnes
        \item Collaboration/entraide très importante en informatique!
        \item Plus on est de fous, plus on rit!
    \end{itemize}
    
    \textbf{En pratique}
    \begin{itemize}
        \item Explication de la théorie avec slides
        \item Petits exercices pour s'entraîner
        \item Corrections et réponses aux questions
        \item Réalisation de projets
    \end{itemize}
\end{frame}

\begin{frame}
\frametitle{Planning} 
\begin{table}[]
	\begin{tabular}{|l|l|l|l|l|}
		\hline
		\rowcolor[HTML]{CA4242} 
		{\color[HTML]{000000} Lundi}                                                           & {\color[HTML]{000000} Mardi}                                       & {\color[HTML]{000000} Mercredi}                                           & {\color[HTML]{000000} Jeudi}                                   & {\color[HTML]{000000} Vendredi} \\ \hline
		\begin{tabular}[c]{@{}l@{}}Introduction au\\ Raspberry,\\ Premier \\programme\end{tabular} & \begin{tabular}[c]{@{}l@{}}Python\\ Types\\ Fonctions\end{tabular} & \begin{tabular}[c]{@{}l@{}}Python \\ Conditions\\ Boucles\end{tabular} & \begin{tabular}[c]{@{}l@{}}Coder un \\jeu: PyGame\end{tabular} & Station Météo                   \\ \hline
	\end{tabular}
\end{table}
    
\end{frame}
    


\section{A propos de vous...} %TODO

\begin{frame}
    \frametitle{A propos de vous...}
    \begin{itemize}
    \item Prénom
    \item Âge et année d'étude
    \item Déjà fait de la programmation? En Python?
    \item Connaissance de Raspberry Pi?
    \item Attentes envers ce stage
    \end{itemize}
\end{frame}


%%%%%%%%%%%%%%%%%%%%%%%%%%%%%%%%%%%%%%%%%%%%%%%%%% SÉPARATION %%%%%%%%%%%%%%%%%%%%%%%%%%%%%%%%%%%%%%%%%%%%%%%%%%


% \begin{frame}
% \frametitle{SÉPARATION}
% \begin{block}{Définition Wikipédia \textit{Intelligence Artificielle}}
% L'intelligence artificielle est la "recherche de moyens susceptibles de doter les systèmes informatiques de capacités intellectuelles comparables à celles des êtres humains".\\
% \end{block}
% \begin{itemize}
% \item Capacité d'une machine à "réfléchir"
% \item Une machine qui exécute des tâches répétitives, pré-programmées peut sembler intelligentes sans pour autant être dotée d'une IA
% \end{itemize}
% \end{frame}

\end{document}