Le Python est interprété, il nous faut donc un interpréteur! Il s'agit simplement d'un programme qui va lire notre code, et exécuter en temps réel les instructions qu'on lui donne.

Théoriquement, c'est le seul outil dont on a réellement besoin. Pratiquement, un développeur s'entoure d'une suite logicielle qui lui rend la vie bien plus facile. On utilisera énormément un \textit{éditeur de code}. C'est un éditeur de texte, à la manière de Microsoft Word, mais spécialisé dans un ou plusieurs langages de programmation. Il va notamment permettre de colorer le code pour le lire plus facilement, indenter automatiquement les lignes, les numéroter, etc.

Dans ce cours, nous allons utiliser l'éditeur de code (Atom) et l'interpréteur séparément.
