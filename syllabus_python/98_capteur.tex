Dans votre matériel, vous disposez d'un capteur d'humidité et de température. Il est possible de le connecter au Raspberry Pi, et de récupérer les données grâce à... Python!

Il faut d'abord installer le module correspondant. L'installation nécessite quelques commandes à taper dans le terminal :\\
\begin{lstlistings}
sudo apt-get update\\
sudo apt-get install build-essential python-dev\\
wget https://github.com/adafruit/Adafruit\_Python\_DHT/archive/master.zip\\
unzip master.zip\\
cd Adafruit\_Python\_DHT-master/\\
sudo python setup.py install\\
\end{lstlistings}

Ensuite, on peut écrire un script en Python pour lire la température et l'humidité.

\begin{python}
# module du senseur
import Adafruit_DHT

# Pin GPIO
pin = 23

# lit les donnees, et les enregistre dans des variables humidity et temperature
humidity, temperature = Adafruit_DHT.read_retry(sensor, pin)

# affiche les donnees, ou un message d'erreur
if humidity is not None and temperature is not None:
    print('Temperature={0:0.1f}*C  Humidite={1:0.1f}%'.format(temperature, humidity))
else:
    print('Erreur de lecture, reessayez!')
\end{python}

C'est juste un exemple d'utilisation du module, mais avec tout ce qu'on a déjà appris, on va pouvoir faire des choses bien plus intéressantes!